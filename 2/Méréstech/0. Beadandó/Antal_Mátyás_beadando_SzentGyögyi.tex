%  <-- ezzel lesznek jelölve a kommentek,
%a legtöbb dolgot jobb békénhagyni úgy ahogy van, általában csak {} közé vagy \section{} alá kell írni majd
%az ITKproc miatt nem kell megformáznod a szöveget tehát ezt mindenképpen hagyjátok a itt, valamint mindig legyen a dokumentum mappájában, hogy elérje a fordító
\documentclass[10pt, conference,a4paper]{ITKproc}

\usepackage[utf8]{inputenc}
\usepackage{graphicx}
\usepackage{gensymb}
\graphicspath{ {images/} }
% correct bad hyphenation here

\hyphenation{pre-sence vi-su-alized si-mu-la-tions mo-le-cu-lar se-ve-ral cha-rac-te-ris-tic CoNSEnsX}


\begin{document}
% ide a {} közé írd a jegyzőkönyv címét
\title{GPS mérési jegyzőkönyv}
% ezek gondolom egyértelműek, itt is mindig csak a {} szerkesszétek, valamint használhattok \\ sortöréshez, pl dátum hozzáírás stb
\author{\IEEEauthorblockN{Mátyás Antal}
\IEEEauthorblockA{(Supervisor: Attila Tihanyi)\\
Pázmány Péter Catholic University, Faculty of Information Technology and Bionics\\
50/a Pr\'ater street, 1083 Budapest, Hungary\\
\texttt{antal.matyas.gergely@hallgato.ppke.hu}}
}


\maketitle

\begin{abstract}
A mérés célja volt ismerkedni a műholdas helymeghatározó rendszer elméletével és működésével. A felkészülés során tanulmányoztuk a GPS és GLONASS rendszerek működését, a GPS-es helymeghatározás módszereit, valamint a VisualGPS program kezelését. 
\end{abstract}

\IEEEpeerreviewmaketitle
% innentől kezdve jönnek a feladatok
% \section{} Ezzel hozunk létre fejezetet, a {} közé pedig bármi írhatunk, általában úgyis "Feladat" és "összefoglalás"-t fogunk,
% a fejezetek alapjáraton számozódnak római számokkal tehát azt nem szükséges beleírni, 
% ha alfejezeteket akarunk létrehozni akkor \subsection{} subsubsection{} stb- vel tegyük
%példa:
\section{dolog}
 	Források:
https://hu.wikipedia.org/wiki/Szent-Gy%C3%B6rgyi_Albert
https://www.kfki.hu/~cheminfo/hun/mvm/arc/szentgy.html


Szent-Györgyi Albert

Született: 1893. szeptember 16-án, Budapesten
Elhunyt: 1988. október 22-én, Woods Hole, Massachusetts

Szülei Szentgyörgyi Miklós és Lenhossék Jozefina. Apai ágon nemesi, anyai ágon pedig orvosprofesszori felmenőkkel jeleskedett. Szülai házasságának megromlása után két testvérével, Pállal és Imrével Pesten élt.
Kezdeti tanulmányait 1904 és 1911 között a Lónyai utcai református gimnáziumban végezte, ahonnan nagybátyja tiltása ellenére a Budapesti Tudományegyetem Orvostudományi Karára jelentkezett.
1914 nyarán a kötelező három hónapos katonai szolgálatát tölti, amikor kitör az első világháború, és a keleti frontra kerül, ahol medikusként tevékenykedik, de elege lesz a háborúból, ezért belelő a karjába, hogy kórházba kerülhessen. Lábadozása alatt folytatja tanulmányait a Tudományegyetemen, és 1917-ben orvosi diplomát szerez.  
1917. szeptember 15-én feleségül veszi Demény Kornéliát, a Magyar Posta vezérigazgatójának lányát. 1918. október 3-án megszületik lánya, Nelli, ennek köszönhetően szabadságot kap, melynek lejárta előtt a háború véget ér.


 A világháború után Pozsonyban, Prágában, Leidenben valamint Gröningben folytat tanulmányokat, ahol eredetileg a holland trópusi orvosi vizsgát szeretné letenni, azonban ennek gyakorlati részén megbukik. Az egyetem professzora, Hartog Jacob Hamburger alkalmazza, vele végez kutatásokat, kutyakísérletekben segédkezik, majd az Addison-kórban szenvedő betegekkel foglalkozik, melyek során 1927-ben felfedez egy a mellékvesében található redukáló hatású anyagot, melyet hexuronsavnak nevez el. Tanulmányozására ösztöndíjat nyer el a Cambridge-i Egyetemre Frederick Hopkins, Nobel-díjas tudós támogatásával. Kutatásai eredménye képpen az egyetem biokémiai tanszékén szerzi meg második doktorátusát, ezúttal kémiából, melyet követően egy évig E.C Kendall támogatásával az Egyesült Államokban dolgozik.

1928-ban Klebelsberg Kuno kultuszminiszter felajánlja neki a Szegedi Tudományegyetem orvosi kémiai tanszékének vezetését.
Hazalátogatása után visszatért Cambridge-be, 1929 nyarán pedig az Amerikai Egyesült Államokban részt vett a Bostonban rendezett élettani világkongresszuson, ahol  Edward C. Kendall meghívta a laboratóriumába, 

1930 augusztusában Szent-Györgyi és családja elhagyta Angliát,majd 1930. október 26-án foglalta el végre szegedi katedráját (és odaköltözött családjával Szegedre) .   1931 januárjától kezdte meg kutatói és tanári tevékenységét az orvosi vegytani intézet professzoraként.

A Rockefeller Alapítvány támogatásával Szent-Györgyi egy modern tudományos központot és biokémiai iskolát hozott létre Szegeden. 
Új oktatási stílust vezetett be, az addig megszokott merev, poroszos, tekintélytiszteletre épülő professzorokkal szemben ő elvárta diákjaitól, hogy vitatkozzanak vele és szabadids programokat is szervezett velükDiákjainak kötelezővé tette a rendszeres sportolást (ő maga is teniszezett, röplabdázott, megtanult vitorlázórepülni, és 1934-ben motorbiciklijével európai körutat tett )

1931 őszén egy fiatal amerikai kutató, Joseph L. Svirbely jelentkezett Szent-Györgyinél, hogy szeretne nála dolgozni egy évvig, amíg magyarországon van.
Svirbely Charles Glen King kutatócsoportjának tagja volt, ahol a C-vitamint próbálták megtalálni, hiszen a skorbutellenes anyag kémiai tulajdonságai még ismeretlenek voltak. Tengerimalacokon folytak a kisérletek.
Szent-Györgyi, a maradék hexuronsavát odaadta a fiatal kutatónak, hogy megnézhesse, tartalmaz- e C- vitamin.
Az első állatkísérlet után kiderült számára, hogy a hexuronsav valójában koncentrált C-vitamin.
1932 márciusában megírta Kingnek  a kisérllet sikerességét, aki ezután április 1-jén a Science magazinban közzétette, hogy megtalálta a C-vitamint.
Hosszú vita követte az eseményeket: mindkét fél ötlete ellopásával vádolta a másikat. 
King 1933-ban megpróbálta szabadalmaztatni a C-vitamint, de a szabadalmi hivatal megállapította Szent-Györgyi elsőbbségét a felfedezésben,
(1932. március 18-án, a Budapesti Királyi Orvosegyesület ülésén bejelentette: "Nyilvánosság előtt először ezúttal mondjuk ki, hogy a hexuronsav és a C-vitamin azonosak." 

Felhasználva összes külföldről hozott hexuronsavát, további kisérletezéseire már nem állt redelkezésére több.
Saját elmondása szerint az egyik vacsorához kapott paprikasalátáját azzal a kifogással érte el, hogy ne kelljen megennie, hogy azt mndenképpen meg kell vizsgálnia a laboratóriumba. Mindenki meglepetésére kiderült, hogy vaóban jelentős mennyiségű C-vitamint tartalmaz- és izolálni is sookkal egyszerűbb volt, mint a citrusféléket.
Összes munkatársa segitségével sikerült egy hét alatt másfél kilónyi C-vitamint kinyerniük.
A C-vitamin nagytömegű előállítása érdekében 1933. szeptemberében egy hold földet kapott bérbe Szeged városától paprika termesztésre -kisérleteihez.
A vegyület nevét is Szent-Györgyi és Haworth adta: aszkorbinsav (skorbutellenes sav)
Ezen felfedezése által vált Szent-Györgyi Albert -egyik pillanatról a másikra- világhírűvé.
1937-ben ítélte neki a Nobel-díj Bizottság az orvosi és fiziológiai díjat a biológiai égéssel kapcsolatos felfedezéseiért-legfőképp a C-vitaminnal és fumársavkatalizátorral végzett kutatómunkájáért.
Ugyan ebben az évben kapott Haworth kémiai Nobel-díjat( a C-vitamin szerkezetének meghatározásáért)
Szent-Györgyi a Nobel-díja mellé járó érmét a téli háború finnországi szenvedőinek ajnlotta fel, ma pedig a Magyar Nemzeti Múzeum tulajdonát képzi.
Szegedi kutatómunkája során más fontos felfedezést is tett. Kmutatta a paprikában -az általa ideiglenesen P-vitaminnak nevezett- flavonoidot (melyszinte minden táplálékban megtalálható)
Egyes szerves savakról felfedezte, hogy katalizátor szerepet játszanak a biológiai oxidációban.
Közel állt a citrátciklus felfedezéséhez, ám a ciklus minden tagját Hans Krebs azonosította.( innen a neve: Szent-Györgyi-Krebs- ciklus)

1938-ban vált a Magyar Tudományos Akadémia rendes tagjává.
1939-ben tanulmányozni kezdte az izomösszehúzódás mechanizmusát.
Egy véletlen esemény következtében fedezték fel az izom egyik alapvető fehérjeösszetevőjét, az aktint. Kimutatták, hogy az aktin és a miozin komplexet alkotnak- ami ATP hozzáadására mozgást végez.(enneek kapcsán jöttek rá, hogy mi okozza a hullamerevséget.)

1944-ben alapító,és igazgatósági tagja volt a Paprikavitamin termelő és értékesítő részvénytársaságnak.

A 2. vh alatt az antifasiszta ellállási mozgalom aktív tagja volt. Illegális csoportot alapított 1942ben. Céljuk egy radikális polgári párt megalapítása volt. 1943 februárjában az ellenzéki pártok támogatásával és a miniszterelnök tudtával Isztanbulba repült, titkos diplomáciai céllal, Magyarország háborúból való kiugrásának előkészítésére. A sikeres tárgyalások ellenére a terv meghiúsult, mivel a németek megtudták Szent-Györgyi küldetésének célját. 1944 márc 19-én, a német megszálláskor illegalitásba kellett vonulnia, először vidéken, majd Budapesten a svéd nagykövetségen bújkált, ahonnan éppen időben menekítették ki, és szovjet befolyás alatt álló területre került, majd a svéd királytól állampolgárságot kapott. 1945-ben a pázmány péter tudományegyetem biokémiai tanszékének vezetője lett. Ugyanekkor nevezték ki az Országos Köznevelési Tanács elnökévé. 1945-ben megalakult a Magyar-Szovjet Művelődési Társaság, melynek díszelnökévé válaszották. A Magyar Természettudományi Akadémiát fizikus barátjával, Baj Zoltánnal közösen hozták létre. Felállította továbbá a magyar penicilinbizottságot, melynek feladata a penicilingyártás tanulmányozása, valamint az országon belüli termelés bevezetése volt.  
1945 ápr 2-án képviselőnek hívták a Parlamentbe. 1945 szeptemberében az Országos Nemzeti Bizottság egyetlen párton kívüli tagja lett. Ráth Istvánnal közösen alapították meg a Servita RT-t, mely először importált penicilint az országba. 1947-ben levelező tagja lett a Szovjet Tudományos Akadémiának. 1948 márc 15-ével Kossuth-díjat kapott. Ekkor már nem élt Magyarországon, így a díjat soha nem vette át. 

Amerikai évei

Ráth Istvánt letartóztatták, ezért úgy döntött, nem tér vissza Magyarországra. 1947 aug 2-án érkezett feleségével New Yorkba, itt alapítja meg a Szent-Györgyi alapítványt, melynek célja fiatal magyar kutatók nyugatra jutását segítette. 

Számos elméleti cikket, valamint ismeretterjesztő könyvet írt. 1954-ben Albert Lasker-díjat kapott. 
1960-as évektől aktív politikai életet él, cikkeket írt és előadásokat tartott a nukleáris fegyverkezési verseny és a Vietnámi háború ellen. 

Élete utolsó két évtizedét a rákkutatásnak szentelte, melyben szerepet játszott a tény, hogy második felesége, valamint lánya is e betegség áldozata lett. 1972-ben létrehozta a Nemzeti Rákkutató Alapítványt. 1973-ban látogatott először újra Magyarországra a Szegedi Biológiai Központ avatására. Itt avatták díszdoktorrá és a Magyarok Világszövetsége tiszteletbeli tagjává. 

Szent-Györgyi 1986 júniusában vese és szívelégtelenség lépett fel, műtéte után néhány hónappal, 1986 október 22-én hunyt el. Az Atlanti-óceán partján álló házában temették el. 
%Ha szeretnéd hogy az adott fejezet ne legyen számozva használj \section*{} -t, pl Acknowledgements


% Az egyenletekre, táblázatokra, listákra stb. itt nem térnék ki, ahhoz mindenképp érdemes kicsit utána olvasni

% A references mindig a legutolsó fejezet lesz

%  minden hivatkozás elnevezünk, ezzel a névvel fogunk hivatkozni a szövegen belül
% \bibitem{} <- hivatkozásnév
% A hivatkozás formája legjobban a példa dokumentumban látszik a tanárúr honlapján, általában: szerző, olvasott anyag neve, közreműködők, hely, év
% a szövegben pedig \cite{Megadott hivatkozásnév} -vel hivatkozunk


% példa:


% Ez a rész mindig marad:---------
\bibliographystyle{ieeetr}
%\bibliography{references}


% that's all folks
\end{document}

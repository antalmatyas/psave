%  <-- ezzel lesznek jelölve a kommentek,
%a legtöbb dolgot jobb békénhagyni úgy ahogy van, általában csak {} közé vagy \section{} alá kell írni majd
%az ITKproc miatt nem kell megformáznod a szöveget tehát ezt mindenképpen hagyjátok a itt, valamint mindig legyen a dokumentum mappájában, hogy elérje a fordító
\documentclass[10pt, conference,a4paper]{ITKproc}

\usepackage[utf8]{inputenc}
\usepackage{graphicx}
% correct bad hyphenation here

\hyphenation{pre-sence vi-su-alized si-mu-la-tions mo-le-cu-lar se-ve-ral cha-rac-te-ris-tic CoNSEnsX}


\begin{document}
% ide a {} közé írd a jegyzőkönyv címét
\title{Feszültségosztó vizsgálata mérési jegyzőkönyv}
% ezek gondolom egyértelműek, itt is mindig csak a {} szerkesszétek, valamint használhattok \\ sortöréshez, pl dátum hozzáírás stb
\author{\IEEEauthorblockN{Mátyás Antal}
\IEEEauthorblockA{(Supervisor:Attila Tihanyi)\\
Pázmány Péter Catholic University, Faculty of Information Technology and Bionics\\
50/a Pr\'ater street, 1083 Budapest, Hungary\\
\texttt{antal.matyas.gergely@hallgato.ppke.hu}}
}


\maketitle

\begin{abstract}
A mérés célja volt megismerkedni az ellenállásokból álló feszültségosztó tulajdonságaival, vizsgálatával, valamint az ELVIS mérőműszer és a hozzá tartotó szoftvercsomag használatával. A mérésre való felkészülés során átismételtük a középiskolában az ellenállásról, feszültségosztóról valamint a Kirchhoff törvényekről tanultakat, továbbá ezek mérési lehetőségeit. 
\end{abstract}

\IEEEpeerreviewmaketitle
% innentől kezdve jönnek a feladatok
% \section{} Ezzel hozunk létre fejezetet, a {} közé pedig bármi írhatunk, általában úgyis "Feladat" és "összefoglalás"-t fogunk,
% a fejezetek alapjáraton számozódnak római számokkal tehát azt nem szükséges beleírni, 
% ha alfejezeteket akarunk létrehozni akkor \subsection{} subsubsection{} stb- vel tegyük
%példa:
\section{Mérendő objektumok}

A mérés kezdetén 3 darab ismeretéen értékű ellenállást kaptunk, első feladatunk tehát ezeknek beazonosítása volt. Mindhárom ellenállás 5 csíkos jelöléssel volt ellátva, az erre vonatkozó táblázat alapján leolvastuk az értéküket, majd az ELVIS digitális multiméter használatával, a V és COM kimenetek közé kapcsolva méréssel ellenőriztük azt. Az $R_1$ ellenállás a jelölés alapján 3300$\Omega$, 1\%-os tűréssel, a mérés eredménye pedig $R_1$ = 3.28$k\Omega$, ami belül esik a tűréshatáron. A második ($R_2$) ellenállás leolvasott értéke 1100$\Omega$ szinén 1\%-os tűréssel, a mért értéke pedig $R_2$ = 1.09$k\Omega$, ami szintén megfelelő érték. A harmadik ($R_3$) ellenállás a színkód alapján 2200$\Omega$, 1\%-os tűréssel, mért értéke pedig $R_3$ = 2.18$k\Omega$. Az ellenállások értéke tehát: \[ R_1 = 3.28k\Omega, R_2 = 1.09k\Omega,   R_3 = 2.18k\Omega\] 
A mérések során minden esetben megmértük a mérőeszközben a nulla ellenállást, illetve feszültséget, és a Null Offset funkció használatával ezt automatikusan kivontuk a később mért értékekből, ezzel kiküszöbölve az offszet hibát. 
\section{Első mérési összeállítás}
\subsection{Elmléleti értékek}
Az ellenállások értékének segítségével kiszámítottuk a névleges feszültségosztási viszonyt, melyhez a \[U_{ki} = U_{be} * \frac{R_1}{R_1+R_2}\] képletet használtuk. A feladat folytatásaképp kiszámítottuk a feszültségosztási viszonyt a tűréshatárok figyelembe vételével a \[U_{be} * \frac{R_1*0.01}{R_1+R_2}\] képlet használatával. A kapott értékeket az I. táblázat szemlélteti. 

\begin{table}[ht!]
\renewcommand{\arraystretch}{1.3}
\caption{Feszültségosztás - elmélet}
\centering
\begin{tabular}{c||c||c}
\hline
\bfseries Ellenállás & \bfseries Feszültség & \bfseries Feszültségosztás maximális eltérés\\
\hline\hline
 $R_1$ = 3.3$k\Omega$ & $U_1$ = 3.75$V$  & $U_{el}$ = 0.0075$V$\\
\hline
 $R_2$ = 1.1$k\Omega$ & $U_2$ = 1.25$V$  & $U_{el}$ = 0.0025$V$\\
\hline
\end{tabular}
\end{table}

\subsection{Számított értékek}
A mérési utasításnak megfelelően ezt követően kiszámítottuk a feszültségosztási viszont, valamint annak maximális eltérését, ezúttal az ellenállások valós (mért) értékét felhasználva. A számítások eredményeit a II. táblázat szemlélteti.

\begin{table}[ht!]
\renewcommand{\arraystretch}{1.3}
\caption{Feszültségosztás - gyakorlati}
\centering
\begin{tabular}{c||c||c}
\hline
\bfseries Ellenállás & \bfseries Feszültség & \bfseries Feszültségosztás maximális eltérés\\
\hline\hline
 $R_1$ = 3.28$k\Omega$ & $U_1$ = 3.73$V$  & $U_{el}$ = 0.0075$V$\\
\hline
 $R_2$ = 1.09$k\Omega$ & $U_2$ = 1.24$V$  & $U_{el}$ = 0.0024$V$\\
\hline
\end{tabular}
\end{table}

\subsection{Mért értékek}
A mérést az utasításban mellékelt ábra alapján összeállítva, gyakorlatban is megvizsgáltuk a feszültségosztást, megmértük értékeit, amik az $U_1 = 3.62 V$, valamint az $U_2 = 1.20 V$ feszültségeket adták, ez megerősíti a korábbi számításainkat.  

\subsection{Tápegység feszültsége}
Ezt követően az ELVIS digtális multiméter segítségével megmértül a tápegység belső feszültségét, majd ennek segítségével korrigáltuk a korábbi számításainkat. A tápegység mért belső feszültsége: $U_{b} = 4.85 V$, ez 3\%-os relatív eltérést jelent. Ez alapján a kimeneti feszültségek: $U_1 = 3.64 V$ és $U_2 = 1.21 V$. Méréssel ellenőriztük a számítások helyességét. 

\section{Második mérési összeállítás}
\subsection{Elméleti értékek}
A második mérési összeállítás elméleti számításai során az $R_2$ és $R_3$ ellenállások eredő ellenállásával számoltunk, mely párhuzamos kapcsolás miatt az $R_{23} = \frac{1}{\frac{1}{R_2} + \frac{1}{R_3}}$ értékkel megegyező. Ezesetben a kimeneti feszültség elméleti értéke: $U_{23} = 0.90 V$. 
\subsection{Mért értékek}
A mérési utasításban mellékelt ábra alapján összeállítottuk a mérést, és az ELVIS digitális multiméter használatával ellenőriztük számításaink helyességét, megmértük a kimeneti feszültség valós értékét, mely az $U_{23} = 0,899 V$ értéket vette fel, mely beleesik az 1\%-os legnagyobb megengedett eltérési értékbe. 

\section{A mérést befolyásoló tényezők}
\subsection{Fesztülséggenerátor}
Ha az alkalmazott feszültséggenerátor belső ellenállása eltér a tervezett értéktől, azaz jelen esetben nem 0, ahogy az egy ideális feszültséggenerátornál volna, úgy a kimenő, valamint minden eső feszültség is változik, hiszen a belső ellenállás hozzáadódik a számítások során használt eredő ellenálláshoz, ezzel módosítva az összeállításban mért áramerősség, valamin így az egyes ellenállásokra eső feszültség értékét is. 
\subsection{Feszültségmérő}
A gyakorlatban alkalmazott feszültségmérő belső ellenállása igen nagy, általában M$\Omega$ nagyságrendű. Amennyiben ez a belső ellenállás csökken, a párhuzamosan kötött feszültségmérőn is áram folyik, így a feszültségosztás módosul, hiszen a mérőműszerre is esik feszültség, ezzel csökkentve az ellenállásokra eső feszültség értékeit. 
\subsection{Befolyásoló tényezők mérése}
\subsubsection{Feszülséggenerátor}

A kimenő feszültség függése a belső ellenállástól az alábbi képlettel szemléltethető: 
\[R_b-\frac{R_b*U_{ki}}{U{be}} = \frac{R_1*U{ki}+R_2*U{ki}}{U_{be}} \]
\[R_b*(U_{be}-U_{ki}) = R_1*U{ki}+R_2*U{ki} \]
\[R_b = \frac{R_1*U{ki} + R_2*U{ki}}{U_{be}-U{ki}} \]

\subsubsection{Feszültségmérő}

A kimenő feszültség függését a bemenő ellenállástól pedig az alábbi képlet írja le: 
\[R_1 = \frac{R_b*U_{ki}+R_2*U{ki}}{U_{be}-U_{ki}} \]
%Ha szeretnéd hogy az adott fejezet ne legyen számozva használj \section*{} -t, pl Acknowledgements


% Az egyenletekre, táblázatokra, listákra stb. itt nem térnék ki, ahhoz mindenképp érdemes kicsit utána olvasni

% A references mindig a legutolsó fejezet lesz

%  minden hivatkozás elnevezünk, ezzel a névvel fogunk hivatkozni a szövegen belül
% \bibitem{} <- hivatkozásnév
% A hivatkozás formája legjobban a példa dokumentumban látszik a tanárúr honlapján, általában: szerző, olvasott anyag neve, közreműködők, hely, év
% a szövegben pedig \cite{Megadott hivatkozásnév} -vel hivatkozunk


% példa:


% Ez a rész mindig marad:---------
\bibliographystyle{ieeetr}
%\bibliography{references}


% that's all folks
\end{document}